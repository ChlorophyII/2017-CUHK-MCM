\documentclass{icmmcm}
\usepackage{graphicx} % For importing graphics.
\usepackage[numbers]{natbib}

%%% Sample ICM/MCM Contest Submission
%%%
% Copyright (C) 2003-2016  Claire M. Connelly

% This work may be distributed and/or modified under the conditions of
% the LaTeX Project Public License, either version 1.3 of this license
% or (at your option) any later version.

% The latest version of this license is in
%   http://www.latex-project.org/lppl.txt
% and version 1.3 or later is part of all distributions of LaTeX
% version 2005/12/01 or later.

% This work has the LPPL maintenance status `maintained'.

% The Current Maintainer of this work is Claire M. Connelly.



%%% ---------------
%%% Local Command and Environment Definitions

%%% If you have any local command or environment definitions, put them
%%% here or in a separate style file that you load with \usepackage.

% \newtheorem declarations
\newtheorem{Theo1}{Theorem}
\newtheorem{Theo2}{Theorem}[section]
\newtheorem{Lemma}[Theo2]{Lemma}
% Each of the above defines a new theorem environment.
% Multiple theorems can be done in the same environment.
% Theo2's number is defined by the subsection it's in.
% Theo3 uses the same numbering counter and numbering system as
% Theo2 (that's the meaning of [Theo2]).


%%% TeX has an excellent hyphenation algorithm, but sometimes it
%%% gets confused and needs some help.
%%%
%%% For words that only occur once or twice, you can insert hints
%%% directly into your text, as in
%%%
%%%    our data\-base system is one of the most complex ever devised
%%%
%%% For words that you use a lot, and that seem to keep ending up at
%%% the end of a line, however, inserting the hints each time gets to
%%% be a drag.  You can use the \hyphenation command  to globally tell
%%% TeX where to hyphenate words it can't figure out on its own.

\hyphenation{white-space}

%%% End Local Command and Environment Definitions
%%% ---------------


%%% ---------------
%%% Title Block

\title{Your Report Title}

%%% Which contest are you taking part in?  (Just one!)

\contest{MCM}

%%% The question you answered.  (Again, just the one.)

\question{C}

%%% Your Contest Team Control Number
\team{63713}


%%% A normal document would specify the author's name (and possibly
%%% their affiliation or other information) in an \author command.
%%% Because the ICM/MCM Contest rules specify that the names of the
%%% team members, their advisor, and their institution should not
%%% appear anywhere in the report, do *not* define an \author command.

%%% Defining the \date command is optional.  If you leave it blank,
%%% your document will include the date that the file is typeset, in
%%% the form  ``Month dd, yyyy''.

% \date{}

%%% End Title Block
%%% ---------------

\begin{document}

%%% ---------------
%%% Summary

\begin{summary}
  The contest rules specify that you should include a one-page summary
  of your report.  This page appears before the rest of the report,
  and will have a special header attached to it that takes up the top
  2.5" of the page.

  By typing your summary inside a \texttt{summary} environment, \TeX\ will
  handle the formatting of that page correctly, including leaving
  space at the top of the page and not numbering the page.
  
  It will also reset the page numbers so that the first page of your
  report is labeled correctly.
  
  What should you put here?  Basically, you want a brief restatement
  of the problem followed by a largely \emph{non-technical}
  description of what you've done.  Try to avoid using mathematical
  notation.
  
  You probably want to write a few paragraphs, around half to
  two-thirds of a page.

  In 2016, the COMAP folks said the following:
  \begin{quotation}
    The summary is an essential part of your MCM/ICM paper. The
    judges place considerable weight on the summary, and winning
    papers are often distinguished from other papers based on the
    quality of the summary.

    To write a good summary, imagine that a reader will choose
    whether to read the body of the paper based on your summary:
    Your concise presentation in the summary should inspire a
    reader to learn about the details of your work. Thus, a
    summary should clearly describe your approach to the problem
    and, most prominently, your most important conclusions.
    Summaries that are mere restatements of the contest problem,
    or are a cut-and-paste boilerplate from the Introduction are
    generally considered to be weak.

    Besides the summary sheet as described each paper should
    contain the following sections:

    \begin{description}
    \item[Restatement and Clarification of the Problem] State in
      your own words what you are going to do.
    \item[Explain Assumptions and Rationale/Justification]
      Emphasize the assumptions that bear on the problem. Clearly
      list all variables used in your model.
    \item[Include Your Model Design and Justification] for type
      model used or developed.
    \item[Describe Model Testing and Sensitivity Analysis],
      including error analysis, etc.
    \item[Discuss the Strengths and Weaknesses] of your model or
      approach.
  \end{description}
 \citep{comap-mcm-rules}
\end{quotation}

\end{summary}
 
%%% End Summary
%%% ---------------

%%% ---------------
%%% Print Title Block, Contents, et al.

\maketitle
\tableofcontents

%%% Uncomment the following lines by deleting the % sign
%%% if you have figures or tables in your report:
% \listoffigures
% \listoftables  
 
%%% End Print Title Block, Contents, et al.
%%% ---------------

\section{Introduction}%
\label{sec:introduction}

The authors intended to model the effect on traffic flow of the percentage of self-driving cars, the average traffic volume, and the number of lanes. In particular, interaction between vehicles are implemented by a function between the relative motion state, namely the relative distance and relative velocity of adjacent vehicles and the acceleration of specific vehicles. Then the model was applied to the data for the roads of interest. Afterwards, the authors attempted to determine an optimal strategy towards the following objectives:

\begin{enumerate}
\item Make sure of the security of this traffic system
\item Maximize the total traffic flow 
\item Avoid sharp change to velocities of vehicles to the greatest extent
\end{enumerate}
In this article, the authors decomposed the problem into the following steps when modeling and simulating:
\begin{enumerate}
\item Construct the model of traffic flow on a transportation network consisting of xxx xx and xxx.
\item Use the model to  evaluate the traffic state of given roads of interest, analyze the equilibria and tipping point, then analysis the policy changes of lanes according to the simulation result.
\item Take sensitive cases such as accidents, vibrations of acceleration of vehicles into consideration.
\end{enumerate}
\begin{quotation}
  In Section~1 we give our definitions and notation. Section~2
  describes our numerical experiments\ldots{}..
  
  We prove our main result, Theorem~6, in Section~5\ldots{}.
\end{quotation}

Of course you would replace the numbers in that example with
appropriate \verb|\ref| commands pointing to the correct
\verb|\label|s in your source.


\section{Basic Assumptions}

Here's where you start to lay things out.


\section{More Important Stuff}

\subsection{Remember to Break Things Up Into Logical Sections}


\section{Conclusion}

Here's your big ending.


%%% ---------------
%%% Bibliography

%\nocite{*}   %%% Include everything in the thesis.bib file.  Be
             %%% careful---some journals and fields expect you to only
             %%% include references for materials that you have
             %%% actually cited in your paper, others allow you to
             %%% include materials you used as ``background'' without
             %%% actually citing specific pages or passages.

%%% Feel free to choose any bibliography style you like.
\bibliographystyle{plainnat}
%%% The filename (without the bib extension) of your bibliography file.
\bibliography{icmmcm}

\end{document}




%%% Local Variables:
%%% mode: latex
%%% TeX-master: t
%%% End:
